%!TEX program = xelatex
\documentclass[aspectratio=169]{ctexbeamer}
% \usepackage{physics}  
                        %%% 宽高比说明 %%%%
%% ctexbeamer宏包支持各种宽高比,但本模板只适配了4:3(默认)和16:9的宽高比背景。
%% 添加选项aspectratio=169或aspectratio=43可以更改宽高比,默认是4:3
\usepackage[bluetheme]{ustcbeamer}
% !TeX root = ./main.tex

% 自适应16:9的宽高比设置
\makeatletter
\@ifclasswith{ctexbeamer}{aspectratio=169}{
		\def\backxscale{4/3}\def\backyscale{1}\def\ustcshift{30}
	}{
		\def\backxscale{1.068}\def\backyscale{1.068}\def\ustcshift{0}
	}
\@ifclasswith{beamer}{aspectratio=169}{
		\def\backxscale{4/3}\def\backyscale{1}\def\ustcshift{30}
	}{
		\def\backxscale{1.068}\def\backyscale{1.068}\def\ustcshift{0}
	}
\makeatother


\newcommand{\maketitleframe}{
	%取消footline,sidebar
	\setbeamertemplate{footline}[footlineoff]
	%%设置本命令以后的背景如下
	\usebackgroundtemplate{%
			% !TeX root = ./main.tex
\begin{tikzpicture}[y=1pt, x=1pt, yscale=0.474*\backyscale, xscale=0.474*\backxscale, inner sep=0pt, outer sep=0pt]
    % \begin{scope}[cm={{1,0.0,0.0,-1,(0,0)}}]
      \path[fill=themecolor1,even odd rule] (720.4059,0.3515) -- (720.4059,71.1507) --
        (0.3436,71.1507) -- (0.3436,0.3515) -- cycle;
    
    
    
      \path[fill=themecolor2,even odd rule] (0.3436,76.0927) -- (720.4059,76.0927) --
        (720.4059,79.3043) -- (0.3436,79.3043) -- cycle;
    
    
    
      \path[left color=themecolor!80, right color=themecolor,even odd rule] (720.4059,392.4714) -- (0.3436,392.4714) --
        (0.3436,112.2780) -- (720.4059,112.2780) -- cycle;
    
    
    
          \path[even odd rule] (720.4059,392.4714) -- (0.3436,392.4714) --
            (0.3436,112.2780) -- (720.4059,112.2780) -- cycle;
    
    
    
      \path[fill=themecolor1,even odd rule] (0.3436,540.3980) -- (720.4059,540.3980) --
        (720.4059,433.5987) -- (0.3436,433.5987) -- cycle;
    
    
    
      \path[fill=themecolor,even odd rule] (84.0129,392.4714) -- (64.1007,392.4714) ..
        controls (139.6636,296.8146) and (237.7485,213.1007) .. (369.3679,152.9909) ..
        controls (258.0055,213.1787) and (160.4523,292.5573) .. (84.0129,392.4714) --
        cycle(0.3436,210.9419) -- (0.3436,112.2780) -- (205.1827,112.2780) .. controls
        (135.2103,142.0132) and (67.0535,175.1304) .. (0.3436,210.9419) --
        cycle(0.3436,318.0849) -- (0.3436,261.2429) .. controls (87.4868,197.7318) and
        (173.6552,147.4217) .. (257.6874,112.2780) -- (329.6364,112.2780) .. controls
        (208.4627,165.2559) and (97.0387,225.2988) .. (0.3436,318.0849) --
        cycle(23.7974,392.4714) -- (0.3436,392.4714) -- (0.3436,375.5798) .. controls
        (103.8628,248.3597) and (243.0859,163.2623) .. (385.7678,112.2780) --
        (411.7601,112.2780) .. controls (266.1313,174.0682) and (125.2848,246.4444) ..
        (23.7974,392.4714);
    
    
    
      \path[fill=themecolor2,even odd rule] (0.3436,428.6568) -- (720.4059,428.6568) --
        (720.4059,425.4452) -- (0.3436,425.4452) -- cycle;

      \node at(185-\ustcshift,488){\textcolor{themecolor}{\includegraphics[scale=1]{./theme/ustc_logo_side.pdf}}};
    % \end{scope}
\end{tikzpicture}
	}%
	%封面页
	\begin{frame}%
		\maketitle%
	\end{frame}%
	%设置本命令以后的背景如下
	\usebackgroundtemplate{%
			% !TeX root = ./main.tex
\begin{tikzpicture}[y=1pt, x=1pt, yscale=0.474*\backyscale, xscale=0.474*\backxscale, inner sep=0pt, outer sep=0pt]
% \begin{scope}[cm={{1,0.0,0.0,-1,(0,0)}}]
    \path[left color=themecolor!80,right color=themecolor,even odd rule] (720.4059,540.3980) -- (0.3436,540.3980) --
      (0.3436,478.8258) -- (720.4059,478.8258) -- cycle;



        \path[even odd rule] (720.4059,540.3980) -- (0.3436,540.3980) --
          (0.3436,478.8258) -- (720.4059,478.8258) -- cycle;



    \path[fill=themecolor,even odd rule] (84.0129,540.3980) -- (64.1007,540.3980) ..
      controls (139.6636,519.3777) and (237.7485,500.9814) .. (369.3679,487.7725) ..
      controls (258.0055,500.9987) and (160.4523,518.4420) .. (84.0129,540.3980) --
      cycle(0.3436,500.5071) -- (0.3436,478.8258) -- (205.1827,478.8258) .. controls
      (135.2103,485.3599) and (67.0535,492.6376) .. (0.3436,500.5071) --
      cycle(0.3436,524.0517) -- (0.3436,511.5606) .. controls (87.4868,497.6042) and
      (173.6552,486.5485) .. (257.6874,478.8258) -- (329.6364,478.8258) .. controls
      (208.4627,490.4677) and (97.0387,503.6621) .. (0.3436,524.0517) --
      cycle(23.7974,540.3980) -- (0.3436,540.3980) -- (0.3436,536.6860) .. controls
      (103.8628,508.7296) and (243.0859,490.0294) .. (385.7678,478.8258) --
      (411.7601,478.8258) .. controls (266.1313,492.4040) and (125.2848,508.3087) ..
      (23.7974,540.3980);

  \path[fill=themecolor2,even odd rule] (720.4059,475.6141) -- (0.3436,475.6141) --
    (0.3436,478.8258) -- (720.4059,478.8258) -- cycle;



  \path[fill=themecolor,even odd rule] (0.3436,0.3515) -- (720.4059,0.3515) --
    (720.4059,28.5100) -- (0.3436,28.5100) -- cycle;



  \path[fill=themecolor2,even odd rule] (720.4059,28.5100) -- (0.3436,28.5100) --
    (0.3436,31.7217) -- (720.4059,31.7217) -- cycle;

  \node at(580+\ustcshift,509){\textcolor{white}{\includegraphics[scale=0.8]{./theme/ustc_logo_side.pdf}}};



% \end{scope}

\end{tikzpicture}
	}%
}%
%
                        %%% ustcbeamer说明 %%%%
%% 宏包使用了TikZ代码形式的背景文件(在子文件夹theme中),默认选项"bluetheme",是科大校徽的蓝色;此外ustcbeamer还内置了红色和黑色主题"redtheme","blacktheme"。

                        %%% 自定义你的主题颜色 %%%
%% 一旦使用了下述命令就会覆盖ustcbeamer的内置颜色选项,你可以设置自己喜欢的RGB色值:
% \definecolor{themecolor}{RGB}{0,150,0} % 这是绿色主题
% \definecolor{themecolor}{RGB}{0,150,150} % 青色主题,也蛮好看的

%% 注意小写rgb和大写RGB表示的色值相差255倍,即RGB{255,255,255}=rgb{1,1,1};
% \definecolor{themecolor}{rgb}{0,0.5,0.3} % 深绿色主题

%% 建议自定义的主题颜色选择偏深色
%%%%%%%%%%%%%%%%%%%%%%%%%%%%%%%%%%%%%%%%%%%%%%%%%%%%%%%%%%%%%%%%%%%%%%


\title[底部简明标题]{
    开放量子体系理论
}
\author[底部演讲者]{报告人:XXX}
\institute[USTC]{
中国科学技术大学,近代物理系
}
\date{\today}


\begin{document}
%\section<⟨mode specification⟩>[⟨short section name⟩]{⟨section name⟩}
%小于等于六个标题为恰当的标题

%--------------------
%标题页
%--------------------
\maketitleframe
%--------------------
%目录页
%--------------------
%beamer 101
\begin{frame}%
	\frametitle{大纲}%
	\tableofcontents[hideallsubsections]%仅显示节
	%\tableofcontents%显示所节和子节
\end{frame}%
%--------------------
%节目录页
%--------------------
\AtBeginSection[]{
\setbeamertemplate{footline}[footlineoff]%取消页脚
  \begin{frame}%
    \frametitle{大纲}
	%\tableofcontents[currentsection,subsectionstyle=show/hide/hide]%高亮当前节,不显示子节
    \tableofcontents[currentsection,subsectionstyle=show/show/hide]%show,shaded,hide
  \end{frame}
\setbeamertemplate{footline}[footlineon]%添加页脚
}
%--------------------
%子节目录页
%--------------------
\AtBeginSubsection[]{
\setbeamertemplate{footline}[footlineoff]%取消页脚
  \begin{frame}%
    \frametitle{大纲}
	%\tableofcontents[currentsection,subsectionstyle=show/hide/hide]%高亮当前节,不显示子节
    \tableofcontents[currentsection,subsectionstyle=show/shaded/hide]%show,shaded,hide
  \end{frame}
\setbeamertemplate{footline}[footlineon]%添加页脚
}

\section{研究背景}
\begin{frame}
  \frametitle{研究背景}
  研究背景:
  \begin{itemize}
    \item 一
    \item 二
    \item 三
  \end{itemize}

\end{frame}


\section{理论模型}
\subsection{模型1}
\begin{frame}
  \frametitle{理论模型1}
  考虑开放系统主方程
  \begin{equation*}
    \frac{d\rho}{d t}=-i \left[H,\rho\right]+\kappa\left(2a\rho a^{\dagger}-a^{\dagger}a\rho-\rho a^{\dagger}a\right)
  \end{equation*}
  \pause
  由此得到……

\end{frame}
\subsection{模型2}
\begin{frame}
  \frametitle{理论模型2}
  近似条件
  \begin{equation*}
    \rho\left(t\right)\approx\rho_{\rm S}\left(t\right)\otimes\rho_{\rm B}
  \end{equation*}
  \pause
  于是……
  \begin{equation*}
    \frac{d\rho}{d t}=\cdots
  \end{equation*}
  
\end{frame}


\section{研究方法}

\begin{frame}
  \frametitle{研究方法}
  \begin{block}{方法一}
    \begin{itemize}
      \item abc
      \item def
    \end{itemize}
  \end{block}
  \pause
  \begin{block}{方法二}
    \begin{itemize}
      \item abc
      \item def
    \end{itemize}
  \end{block}
\end{frame}


\section{总结展望}

\begin{frame}
  \frametitle{总结展望}
  \begin{columns}
    \begin{column}{0.50\textwidth}
      \begin{figure}
        \includegraphics[width=0.8\textwidth]{figures/ustc_logo.pdf}
        \caption{标题}
      \end{figure}
    \end{column}
    \begin{column}{0.50\textwidth}
      \begin{block}{结论}
        \begin{itemize}
          \item 结论 1
          \item 结论 2
          \item 结论 3
        \end{itemize}
      \end{block}
    \end{column}
  \end{columns}
\end{frame}

\begin{frame}
  \frametitle{致谢}
  \centerline{\Large 谢谢!}
\end{frame}

\end{document}
