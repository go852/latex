%!TEX program = xelatex
\documentclass[aspectratio=169]{ctexbeamer}
\usepackage{ctex}
\begin{document}
\title[底部简明标题]{
代数习题讲解
}
\author[底部演讲者]{报告人:丁保华}
\begin{frame}
斐波那契数列的发明者是意大利数学家‌列昂纳多·斐波那契(Leonardo Fibonacci),他生于公元1170年,卒于1240年(或1250年),籍贯是比萨。斐波那契在1202年撰写的《‌珠算原理》(Liber Abacci)一书中首次引入了这一数列,用以描述兔子繁殖的问题,因此斐波那契数列也被称为“兔子数列”。
\end{frame}

\begin{frame}
\begin{quote}
找规律
\end{quote}
\[
1,  2, 3, 5, \dots,
\]
\end{frame}

\end{document}