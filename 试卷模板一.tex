\documentclass[UTF8,twoside,openright]{ctexbook}
\usepackage{titlesec} 
\titleformat{\chapter}{\centering\Huge\bfseries}{第\,\chinese{chapter}\,章}{1em}{}
\titleformat{\section}[block]{\centering\Large\bfseries}{第 ~\thesection~讲}{1em}{}[]

%\definecolor{mybackground}{rgb}{0.99, 0.91, 0.95} % 淡紫
\renewcommand*{\thesection}{\arabic{section}}
\usepackage{tasks}%选择题宏包,tasks环境
% texlive 2021 用以下设置
%=====================================
\settasks{                           %
    label=\Alph*.,                   %
    label-offset={0.4em},            %
    label-align=left,                %
    column-sep={10pt},               %
    label-width=2ex,                 %
    item-indent={40pt},              %
    before-skip={-0.7em},            %
    after-skip={-0.7em}              %
    }                                %
%=====================================

%============================================会让\begin{enumerate}的子集列表间距收缩
\usepackage{color}
\usepackage[dvipsnames,svgnames]{xcolor}
\usepackage{enumerate}
% list/itemize/enumerate setting
\usepackage[shortlabels,inline]{enumitem}
\setlist{nolistsep} 

\usepackage{textcircle-cid}
\newfontfamily\koz{KozMinPr6N-Regular.otf}
\newcommand{\quan}[1]{\raisebox{-0.05em}{\koz\CIDtextcircled{#1}}}
\newcommand{\hquan}[1]{\raisebox{-0.05em}{\koz\CIDtextblackcircled{#1}}}
\newcommand{\fquan}[1]{\raisebox{-0.05em}{\koz\CIDtextboxed{#1}}}
\newcommand{\hfquan}[1]{\raisebox{-0.05em}{\koz\CIDtextblackboxed{#1}}}
\usepackage{etoolbox}
\robustify{\quan}
\robustify{\hquan}
\robustify{\fquan}
\robustify{\hfquan}
\newenvironment{zuoye}{
\begin{enumerate}
[
labelindent=0mm,
align=left,
labelwidth=1.5em,
itemsep=2em,
labelsep=0em,
leftmargin=1.5em,
label={\textcolor{cyan}{\hfquan{\arabic*}}}]
}
{\end{enumerate}}

\usepackage{geometry}
\geometry{a4paper,top=3cm,bottom=2.5cm,left=2cm,right=2cm} 

\begin{document}
\definecolor{mybackground}{rgb}{0.99, 0.91, 0.95} % 淡紫
\pagecolor{mybackground}

\chapter{集合的章节}

\section{集合的基本概念}

\begin{zuoye}

    \item 已知函数~$f(x)=a(x-1)-\sin x(a>0)$ 恰有两个零点~$x_{1} $,$x_{2} $,且~$x_{1} <x_{2} $,则~$x_{1} -\tan x_{1} =$~(\quad)
\begin{tasks}(4)
    \task $2$ \task $-2$ \task $-1$ \task $1$
\end{tasks}

\item 设函数$f(x)=
\begin{cases}
{2^{x},x\leq 0} \\
 {\log_{2} x,x>0} \\
\end{cases}
$,函数$y=f(f(x))-1$的零点个数为~(\quad)
\begin{tasks}(4)
    \task $3$ \task $4$ \task $6$ \task $12$
\end{tasks}

\item 已知实数~$a$,$b$ 满足~$2^{a}=3$,$3^{b}=2$,则函数~$f(x)=a^{x}+x-b$ 的零点所在的区间是~(\quad)
\begin{tasks}(4)
    \task $(-2, -1)$ \task $(-1,0)$ \task $(0,1)$ \task $(1,2)$
\end{tasks}

\item 函数~$f(x)=mx^{2}-2x+1$ 有且仅有一个正实数零点,则实数~$m$ 的取值范围是~(\quad)
\begin{tasks}(4)
    \task $(-\infty, 1]$ \task $(-\infty, 0] \cup \{1\}$ \task $(-\infty,0)\cup (0,1]$ \task $(-\infty,1)$
\end{tasks}

\item 若~$f(x)$ 为奇函数,且~$x_{0} $ 是~$y=f(x)-e^{x}$ 的一个零点,则~$-x_{0}
$ 一定是下列哪个函数的零点~(\quad)
\begin{tasks}(2)
    \task $y=f(x)e^{x}+1$ \task $y=f(-x)e^{-x}-1$
    \task $y=f(x)e^{x}-1$ \task $y=f(-x)e^{-x}+1$
\end{tasks}

\item 已知二次函数~$f(x)=x^{2}+bx+c$ 的两个零点分别在区间~$(-2,-1)$ 和~$(-1,0)$ 内,则~$f(3)$ 的取值范围是~(\quad)
\begin{tasks}(4)
    \task $(12,20)$ \task $(12,18)$ \task $(18,20)$ \task $(8,18)$
\end{tasks}

\item 已知函数~$f(x)=\log_{2}(ax^{2}+2x+3)$,若对于任意实数~$k$,总存在实数~$x_{0} $,使得~$f(x_{0}
)=k$ 成立,则实数~$a$ 的取值范围是~(\quad)
\begin{tasks}(2)
    \task $[-1.\frac{1}{3})$ \task $[0,\frac{1}{3}]$
    \task $[3,+\infty )$ \task $(-1,+\infty)$
\end{tasks}

\item 已知函数~$f(x)=|\ln x|-a^{x}(x>0$,$0<a<1)$ 的两个零点是~$x_{1} $,$x_{2}
$,则~(\quad)
\begin{tasks}(4)
    \task $0<x_{1} x_{2} <1$ \task $x_{1} x_{2} =1$
    \task $1<x_{1} x_{2} <e$ \task $x_{1}x_{2} >e$
\end{tasks}


\item 已知函数~$f(x)=e^{x-1}+4x-4$,$g(x)=\ln x-\frac{1}{x}$,若~$f(x_{1}
)=g(x_{2} )=0$,则~(\quad)
\begin{tasks}(2)
    \task $0<g(x_{1} )<f(x_{2} )$ \task $f(x_{2} )<g(x_{1} )<0$
    \task $f(x_{2} )<0<g(x_{1} )$ \task $g(x_{1} )<0<f(x_{2} )$
\end{tasks}

\item 已知函数$f(x)=
\begin{cases}
{x^{2}+4x,x\leq 0} \\
 {x\ln x,x>0} \\
\end{cases}
$,$g(x)=kx-1$,若方程~$f(x)-g(x)=0$在$x\in
(-$2,$e)$ 时有~3 个实数根,则~$k$ 的取值范围为~(\quad)
\begin{tasks}(2)
    \task $(1,1+\frac{1}{e})\cup [\frac{3}{2},2) $
    \task $(1+\frac{1}{e},\frac{3}{2}) $
    \task $(\frac{3}{2},2) $
    \task $(1,1+\frac{1}{e})\cup (\frac{3}{2},2)$
\end{tasks}

\item 函数$f(x)=\frac{2}{x-1}+\ln \frac{1}{x}$的零点所在的大致区间是~(\quad)
\begin{tasks}(4)
    \task $(0, 1) $ \task $ (1, 2)$
    \task $(2, 3) $ \task $(0, 1),(2, 3) $
\end{tasks}

\item 函数~$f(x)=2x^{6}-x^{4}-1$ 的零点个数是~(\quad)
\begin{tasks}(4)
    \task $4$ \task $2$
    \task $1$ \task $0$
\end{tasks}

\item 已知~$[x]$ 表示不超过$x$的最大整数,当~$x\in
\mathbf{R}$ 时,称~$y=[x]$ 为取整函数,例如~$[1.6]=1$,$[-3.3]=-4$,若~$f(x)=[x]$,$g(x)$ 的图象关于~$y$ 轴对称,且当~$x\leq
0$ 时,$g(x)=-x^{2}-2x$,则方程~$f(f(x))=g(x)$ 解的个数为~(\quad)
\begin{tasks}(4)
    \task $1 $ \task $2$
    \task $3$ \task $4$
\end{tasks}

\item 已知函数~$f(x)=
\begin{cases}
 {|\log_{3} x|,0<x\leq 3} \\
 {\frac{1}{3}x^{2}-\frac{10}{3}x+8,x>3} \\
\end{cases}
$,若方程~$f(x)=m(m\in
\textbf{R})$ 有四个不同的实根~$x_{1} $,$x_{2} $,$x_{3} $,$x_{4}
$,且满足~$x_{1} <x_{2} <x_{3} <x_{4} $,则~$\frac{(x_{3} -3)(x_{4} -3)}{x_{1}
x_{2} }$ 的取值范围是~(\quad)
\begin{tasks}(4)
    \task $(0,4]$ \task $(0,3) $
    \task $ (3,4]$ \task $(1,3) $
\end{tasks}

\item 已知定义在~$\mathbf{R}$ 上的偶函数~$f(x)$ 满足~$f(x)-f(2-x)=0$,且当~$x\in
[0,1]$ 时,有~$f(x)=-x^{2}+1$,若函数~$g(x)=f(x)-\log_{a}
(2|x|+1)$($a>0$,且~$a\ne 1$) 存在~10 个零点,则实数~$a$ 的取值范围是~(\quad)
\begin{tasks}(4)
    \task $(7,11)$ \task $(9,13)$
    \task $(7,13)$ \task $(9,11)$
\end{tasks}

\item 已知~$f(x)=|xe^{x}{\rm |}$,又$g(x)=[f(x)]^{2}+tf(x)(t\in
\textbf{R}$),若关于~$x$ 的方程~$g(x)=-1$ 有四个不同的实根,则实数$t$的取值范围为~(\quad)
\begin{tasks}(2)
    \task $(-\infty,-\frac{e^{2}+1}{e}) $
    \task $(\frac{e^{2}+1}{e},+\infty) $
    \task $ (-\frac{e^{2}+1}{e},-2)$
    \task $(2,\frac{e^{2}+1}{e})$
\end{tasks}

\item 已知函数~$f(x)=
\begin{cases}
{(x-2)\times |2^{x}-1|,x<2} \\
 {3-\frac{3}{x-1},x>2} \\
\end{cases}
 $,若函数~$g(x)=f(x)-mx+2m$ 有三个不同的零点,则实数~$m$ 的取值范围为~(\quad)
 \begin{tasks}(4)
    \task $ (-1,0)$ \task $(0,1)$
    \task $(-1,1)$ \task $(1,3)$
\end{tasks}

\item 定义函数~$f(x)=
\begin{cases}
 {4-8|x-\frac{3}{2}|,1\leq x\leq 2}  \\
 {\frac{1}{2}f(\frac{x}{2}),x>2}  \\
\end{cases}
$,则函数~$g(x)=xf(x)-6$ 在区间~$[1, 2^{n}](n\in
\mathbf{N}^{\ast})$ 内所有零点的和为~(\quad)
\begin{tasks}(4)
    \task $ n$
    \task $2n$
    \task $ \frac{3}{4}(2^{n}-1)$
    \task $\frac{3}{2}(2^{n}-1) $
\end{tasks}

\item 已知函数~$f(x)=\log_{3}
(a-3^{x})+x-2$,若~$f(x)$ 存在零点,则实数$a$的取值范围
是~\underline{\qquad}

\item 已知函数~$f(x)=(x+1)e^{x}-ax$ 在区间~$[-3,-1]$ 上为减函数,当实数~$a$ 取最小值时,函数~$f(x)$ 的零点个数为\underline{\qquad}.

\item 已知函数~$f(x)=x^{2}-2x\sin \frac{\pi
}{2}x+1$ 的两个零点分别为~$a$,$b$,则~$a+b=$\underline{\qquad}.

\item 已知~$f(x)=x\ln x-\frac{k}{x}(k\in
\mathbf{R})$,其图象与~$x$ 轴交于不同的两点~$A(x_{1} ,0)$,$B(x_{2},0)$,且~$x_{1} <x_{2}$.

(1)求实数$k$的取值范围;

(2)证明:$x_{1} +x_{2} <\frac{2}{\sqrt{{\rm e}}}$.

\item 已知函数~$f(x)=ax^{2}-bx+\ln x$,$a$,$b\in \mathbf{R}$.

(1)当$b=2a+$1时,讨论函数$f(x)$的单调性;

(2)当$a=$1,$b>$3时,记函数$f(x)$的导函数${f}'(x)$的两个零点分别是~$x_{1}
$和~$x_{2} (x_{1} <x_{2} )$,求证:$f(x_{1} )-f(x_{2}
)>\frac{3}{4}-\ln 2$.

%\end{enumerate}
\end{zuoye}
\end{document}